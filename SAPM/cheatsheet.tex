\documentclass[10pt,landscape]{article}
\usepackage{multicol}
\usepackage{calc}
\usepackage{ifthen}
\usepackage[landscape]{geometry}
\usepackage{amsmath,amsthm,amsfonts,amssymb}
\usepackage{color,graphicx,overpic}
\usepackage{hyperref}
\usepackage{booktabs}
\usepackage{adjustbox}


\pdfinfo{%
  /Title (example.pdf)
  /Creator (TeX)
  /Producer (pdfTeX 1.40.0)
  /Author (Seamus)
  /Subject (Example)
  /Keywords (pdflatex, latex,pdftex,tex)}

% This sets page margins to .5 inch if using letter paper, and to 1cm
% if using A4 paper. (This probably isn't strictly necessary.)
% If using another size paper, use default 1cm margins.
\ifthenelse{\lengthtest{ \paperwidth = 11in}}
    { \geometry{top=.5in,left=.5in,right=.5in,bottom=.5in} }
    {\ifthenelse{ \lengthtest{ \paperwidth = 297mm}}
        {\geometry{top=1cm,left=1cm,right=1cm,bottom=1cm} }
        {\geometry{top=1cm,left=1cm,right=1cm,bottom=1cm} }
    }
\newcommand{\printvalues}{topsep=\the\topsep; itemsep=\the\itemsep; parsep=\the\parsep; partopsep=\the\partopsep}

% Turn off header and footer
\pagestyle{empty}

% Redefine section commands to use less space
\makeatletter
\renewcommand{\section}{\@startsection{section}{1}{0mm}%
                                {-1ex plus -.5ex minus -.2ex}%
                                {0.5ex plus .2ex}%x
                                {\normalfont\large\bfseries}}
\renewcommand{\subsection}{\@startsection{subsection}{2}{0mm}%
                                {-1explus -.5ex minus -.2ex}%
                                {0.5ex plus .2ex}%
                                {\normalfont\normalsize\bfseries}}
\renewcommand{\subsubsection}{\@startsection{subsubsection}{3}{0mm}%
                                {-1ex plus -.5ex minus -.2ex}%
                                {1ex plus .2ex}%
                                {\normalfont\small\bfseries}}
\makeatother

% Define BibTeX command
\def\BibTeX{{\rm B\kern-.05em{\sc i\kern-.025em b}\kern-.08em
    T\kern-.1667em\lower.7ex\hbox{E}\kern-.125emX}}

% Don't print section numbers
\setcounter{secnumdepth}{0}


\setlength{\parindent}{0pt}
\setlength{\parskip}{0pt plus 0.5ex}

%My Environments
\newtheorem{example}[section]{Example}
% -----------------------------------------------------------------------

\begin{document}
\raggedright
\footnotesize
\begin{multicols}{2}


% multicol parameters
% These lengths are set only within the two main columns
%\setlength{\columnseprule}{0.25pt}
\setlength{\premulticols}{1pt}
\setlength{\postmulticols}{1pt}
\setlength{\multicolsep}{1pt}
\setlength{\columnsep}{2pt}

\section{Common Quality Attributes}
\begin{tabular}{l l}
  Availability & Need to be there when required i.e.\ can always get a call through \\
  \midrule
  Performance & Control resource demand tactics i.e.\ robot voice doesn't start mid call \\
  \midrule
  Modifiability & Can change an update system i.e.\ want to add video calls \\
  \midrule
  Reliability & System does not fail i.e.\ call does not drop \\
  \midrule
  Security & System is not vulnerable i.e.\ people can't listen in \\
  \midrule
  Testability & Possible to test in any way required \\
\end{tabular}

\section{Common Stakeholders}
\begin{itemize}
    \setlength{\itemsep}{0.05em}
  \item User
  \item Service Provider
  \item Analysis
  \item Developer
\end{itemize}

\section{Common Design Patterns}
\begin{tabular}{l l}
  Layered & distinct layers, can only interact with adjacent layers \\
  \midrule
  MVC & UI and application isolation, user interface changes often \\
  \midrule
  Blackboard & think /r/place \\
  \midrule
  N-tier Architecture & See Layered \\
\end{tabular}

\section{Types of Structures}
\begin{tabular}{l l}
  Modular & Static structures, focus on how functionality is divided up \\
  \midrule
  Component \& Connector & Runtime structures focused on interactions \\
  \midrule
  Allocation & Mapping to Environments \\
\end{tabular}

\section{Scenarios}
\begin{tabular}{l l}
  \toprule
  \textbf{Portion of Scenario} & \textbf{Possible Values} \\
  \midrule
  Source & Internal or external to the system (e.g.\ user) \\
  Stimulus & Arrival of a periodic, sporadic or stochastic event \\
  Artefact & System or one or more components in the system \\
  Environment & Operational mode: normal, emergency, peak overload, overload \\
  Response & Process events, change level of service \\
  Response Measure & Latency, deadline, throughput, jitter, miss rate \\
  \bottomrule
\end{tabular}

\section{Availability}
\subsection{Acronyms \& Measures}
\begin{itemize}
  \setlength{\itemsep}{0.05em}
  \item mtbf --- mean time between failures
  \item mttr --- mean time to repair
  \item usually defined to be the probability it will be there when you ask it to work:
    \[\frac{mtbf}{mtbf + mttr}\]
\end{itemize}

\subsection{Availability Tactics}
\subsubsection{Fault Detection}
\begin{itemize}
    \setlength\itemsep{0.1em}
  \item Ping/Echo;
  \item System monitor (watchdog/heartbeat);
  \item Exception detection (system detections);
  \item Voting (triple modular redundancy)
\end{itemize}

\subsubsection{Fault Recovery}

\subsubsection{\textit{Preparation \& Recovery}}
\begin{itemize}
    \setlength\itemsep{0.1em}
 \item Active/Passive Redundancy;
 \item Spare;
 \item Exception handling (error codes, exception classes);
 \item Software upgrade
\end{itemize}

\subsubsection{\textit{Reintroduction}}
\begin{itemize}
    \setlength\itemsep{0.1em}
 \item Shadow;
 \item State Resynchronisation (graceful restart);
 \item Rollback (un/co-ordinated checkpointing);
 \item Escalating restart;
 \item Non-stop forwarding
\end{itemize}

\subsubsection{Fault Prevention}
\begin{itemize}
    \setlength\itemsep{0.1em}
 \item Removal from service;
 \item Transactions (ACID);
 \item Process monitor
 \item Exception prevention (exception classes; smart pointers; wrappers)
\end{itemize}

\section{Performance Tactics}
\subsection{Control Resource Demand}
\begin{itemize}
    \setlength\itemsep{0.1em}
  \item Manage sampling rate
  \item Limit event response
  \item Prioritise events (priority queues)
  \item Reduce overhead
  \item Bound execution times (timeouts)
  \item Increase resource efficiency (CDNs?)
\end{itemize}

\subsection{Manage Resources}
\begin{itemize}
    \setlength\itemsep{0.1em}
  \item Increase resources (vertical scaling)
  \item Introduce concurrency (horizontal scaling)
  \item Replication
  \item Bound queue sizes
  \item Schedule resources (auto-scaling?)
\end{itemize}

\section{Modifiability Tactics}
\subsection{Reduce Size of a Module}
\begin{itemize}
  \item Split a Responsibility (Single Responsibility principle)
\end{itemize}

\subsection{Increase Cohesion}
\begin{itemize}
  \item Maintain semantic coherence;
  \item Abstract common services
\end{itemize}

\subsection{Reduce Coupling}
\begin{itemize}
  \item Use encapsulation, intermediaries \& wrappers;
  \item Raise abstraction level;
  \item Restrict communication paths
\end{itemize}

\subsection{Defer Binding}
\begin{itemize}
  \item Use runtime registration;
  \item Use start-up time binding; 
  \item Use runtime binding 
\end{itemize}

\section{Security Tactics}
\subsection{Detect Attacks}
\begin{itemize}
  \item Detect intrusion (IDS)
  \item Detect service denial (DoS)
  \item Verify message integrity (checksums?)
  \item Detect message delay
\end{itemize}

\subsection{Resist Attacks}%
\label{sub:resist_attacks}
\begin{itemize}
  \item Identify actors (Accountability)
  \item Authenticate actors
  \item Authorise actors
  \item Limit access
  \item Limit exposure
  \item Encrypt data
  \item Separate entities
  \item Change default settings
\end{itemize}

\subsection{React to Attacks}%
\label{sub:react_to_attacks}
\begin{itemize}
  \item Revoke access
  \item Lock computer
  \item Inform actors
\end{itemize}

\subsection{Recover from Attacks}%
\label{sub:recover_from_attacks}
\begin{itemize}
  \item Maintain audit trail
  \item Restore (See Availability tactics)
\end{itemize}

\section{Testability Tactics}%
\label{sec:testability_tactics}

\subsection{Control and Observe System State}%
\label{sub:control_and_observe_system_state}
\begin{itemize}
  \item Specialised interfaces
  \item Record/playback
  \item Localise state storage
  \item Abstract data sources
  \item Sandbox
  \item Executable assertions
\end{itemize}

\subsection{Limit Complexity}%
\label{sub:limit_complexity}
\begin{itemize}
  \item Limit structural complexity
  \item Limit non-determinism
\end{itemize}

\section{Lifecycles}
\begin{itemize}
    \setlength\itemsep{0.1em}
  \item V Model
    \begin{itemize}
        \setlength\itemsep{0.1em}
      \item ``Verification \& Validation'' model
      \item Extension of waterfall model
      \item Testing planned in parallel with development
      \item Coding phase joins these two sides
    \end{itemize}
  \item RUP
    \begin{itemize}
        \setlength\itemsep{0.1em}
      \item ``Rational Unified Process''
      \item Kind of like a Gantt chart
    \end{itemize}
  \item Spiral Model
    \begin{itemize}
        \setlength\itemsep{0.1em}
      \item Risk driven process model
      \item Iterative design model
    \end{itemize}
  \item Agile SCRUM
    \begin{itemize}
        \setlength\itemsep{0.1em}
      \item 2--4 week sprint
      \item Daily scrum meetings
      \item Most important issues tackled first as they are on top of backlog
    \end{itemize}
\end{itemize}
%\section{QA Testing Senarios}
%\begin{itemize}
%\item Availability
%\item Performance
%\item Modifiability
%\item Reliability
%\item Security
%\item Testability
%\end{itemize}
% You can even have references
\end{multicols}
\end{document}
